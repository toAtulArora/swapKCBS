%% LyX 2.3.5.2 created this file.  For more info, see http://www.lyx.org/.
%% Do not edit unless you really know what you are doing.
\documentclass[british]{article}
\usepackage[T1]{fontenc}
\usepackage[latin9]{inputenc}
\usepackage{geometry}
\geometry{verbose,lmargin=1cm,rmargin=1cm}
\usepackage{verbatim}
\usepackage{amsmath}
\usepackage{amsthm}
\usepackage{amssymb}

\makeatletter
%%%%%%%%%%%%%%%%%%%%%%%%%%%%%% Textclass specific LaTeX commands.
\numberwithin{equation}{section}
\numberwithin{figure}{section}

\makeatother

\usepackage{babel}
\begin{document}
\title{Local and Robust Self testing using trapped ions}
\author{Xiaomin etc.., Atul Singh Arora, Kishor Bharti, Adan Cabello, Kwek
Leong Chuan}
\maketitle
\begin{abstract}
We provide experimental implementation of KCBS self-testing scheme.
Theoretical tools are supplemented to render the results in \textit{Physical
Review Letters 122 (25), 250403 }practical.
\end{abstract}

\section{Introduction}

\subsection{Why self-testing}

Understanding the internal workings of quantum devices becomes challenging
and crucial with the increase in their corresponding dimensionality.
The task of quantum device certification is inherently demanding due
to the exponential scaling of Hilbert space as system size increases.
In a faithfully minimalistic scenario, developing confidence in the
inner functioning of quantum devices mandates certification schemes
that require minimal assumptions regarding their functioning. One
of the most potent methodologies to render guarantees regarding the
internal functioning of quantum devices solely based on experimental
statistics, under reasonable assumptions, is self-testing. The first
self-testing protocol for a pair of non-communicating entangled quantum
devices was devised in the milestone paper by Mayers and Yao in 2004,
which relied on Bell nonlocality. Since then, self-testing via Bell
nonlocality has been exhibited for GHZ states, all pure bipartite
entangled states, and all multipartite entangled states that admit
a Schmidt decomposition. The idea of self-testing has been further
extended to the prepare-and-measure scenario, contextuality and quantum
steering. For a comprehensive review of self-testing, we refer to
{[}cite{]}.

\begin{comment}

\subsection{State of the art}
\begin{itemize}
\item Bell/Device independence | Tensor structure
\begin{itemize}
\item Limitation: Difficult to enforce etc
\item Curve less robust to noise
\item Math is harder
\end{itemize}
\item Semi device independent
\begin{itemize}
\item Easier to implement
\item EPR steering {[}K{]}; almost like blind computing
\end{itemize}
\item All bipartite entangled states can be self-tested by quantum steering
| EPR steering
\end{itemize}
\end{comment}


\subsection{Local Self testing }

Self-testing based on Bell nonlocality is a powerful method which
allows us to acquire insights concerning the inner workings of a duo
of space-like separated entangled devices. However, for single untrusted
quantum devices, self-testing based on Bell-nonlocality forfeits its
relevance, and one requires local self-testing schemes for such cases.
Since Computation typically happens locally, a quantum computer is
a canonical example of a single quantum device. This mandates the
development of self-testing protocols for single untrusted quantum
devices. The first local self-testing scheme was presented in {[}PRL{]}
where the scheme relied on violation of non-contextuality inequalities
via qutrits. Non-contextuality inequalities are linear inequalities,
similar to Bell inequality, the violation of which can be used to
witness non-classicality. The advanatage of non-contextuality inequalities
over Bell inequality is that it can be implemented for a single device
and thus bypasses the non-communication assumption. In a subsequent
work {[}arXiv{]}, the aforementioned scheme was generalized to arbitrarily
high dimension. Using complexity theoretic arguments, self-testing
scheme for a single computationally bounded quantum device was given
in {[}Vidick{]}. For a self-testing protocol to be of experimental
relevance, the scheme must be robust against experimental noise. Though
the self-testing schemes in {[}ADD{]} are robust, the robustness was
proven upto multiplicative constants and hence is not directly implementable.

\subsection{Contextuality}

Informal quick introduction

\subsubsection{Contribution}

Address limitations of Kishor et al's article
\begin{itemize}
\item theoretically obtained the robustness curve
\begin{itemize}
\item Moment matrices based approach
\item First robustness curve for a non-contextuality inequality (five cycles
no less!)
\end{itemize}
\item experimental demonstration of local self-testing
\end{itemize}

\subsubsection{Assumptions and Memory leakage :P}

Self testing works under the following.

\subsubsection{Circumventing the memory assumption}
\begin{itemize}
\item Memory based
\begin{itemize}
\item Quantum classical gap?
\end{itemize}
\item Dimensions based measures
\begin{itemize}
\item Quantum classical gap 
\item log{[}classical dimensions{]}=memory; quantum dimension = quantum
memory
\end{itemize}
\end{itemize}
Criticism:

BQP$\subseteq$PSPACE; that's fine we hope to find the exact curve
to certify
\begin{itemize}
\item Poly separation or some such.
\end{itemize}

\section{Background}

\subsection{Exclusivity graph approach to contextuality}

\subsection{Robust self-testing}
\begin{itemize}
\item Definition
\item Statement for KCBS {[}PRL{]}
\item Robustness curve is missing! Or is it?
\end{itemize}


\section{Robustness Curve}

\subsection{Description}
\begin{itemize}
\item General description (explain importance)
\begin{itemize}
\item Consider any state and measurement system, which we henceforth call
a configuration, that is consistent with the observed data. A robustness
curve is a lower bound on the fidelity of any such configuration with
the ideal configuration. The curve is a relation between the fidelity
and the value of the KCBS expression.
\item More formally, this means $\text{min}\mathcal{F}\left(\left(\left|\psi^{\text{ideal}}\right\rangle ,\left\{ \Pi_{i}^{\text{ideal}}\right\} _{i=1}^{5}\right),\left(\left|\psi^{\text{cand}}\right\rangle ,\left\{ \Pi_{i}^{\text{cand}}\right\} _{i=1}^{5}\right)\right)$
subject to the constraint that $\left|\psi^{\text{cand}}\right\rangle ,\left\{ \Pi_{i}^{\text{cand}}\right\} _{i=1}^{5}$
satisfy the experimental statistics.
\item Fidelity= fidelity of state plus fidelity of measurement
\item Finding a lower-bound is non-trivial.
\end{itemize}
\end{itemize}

\subsection{Overall Strategy {[}TODO: update later{]}}
\begin{itemize}
\item Regularizing the experimental statistics
\item Constructing the isometry
\item Deriving the expression for fidelity
\item Providing the lower bound
\end{itemize}
The goal is to provide a lower bound on fidelity for the experimental
results in section 2. The following steps are required:

\subsection{Regularizing the experimental statistics}

In reference \cite{Bharti2018}, authors assume that the measurement
operators follow cyclic compatibility structure. However, this need
not be true in the real life experimental scenario. Thus, the experimental
data may not belong to quantum set due to finite statistics and violation
of the cyclic compatibility structure. This demands a regularization
of the experimental data for the theoretical analysis. Here, regularization
means finding the statistics which is closest to our experimental
statistics and belongs to the quantum set. The set of quantum behaviours
forms a convex set, known as theta body. 

\subsection{Providing a lower bound on the fidelity from regularized measurement
statistics}

\subsubsection{Constructing the isometry}

A swap circuit between two qudit registers is given by $S=TUVU$ where

\[
T=\mathbb{I}\otimes\sum_{k}\vert-k\rangle\langle k\vert
\]

\[
U=\sum_{k=0}^{d-1}P^{k}\otimes\vert k\rangle\langle k\vert
\]

\[
V=\sum_{k=0}^{d-1}\vert k\rangle\langle k\vert\otimes P^{-k}
\]

and

\[
P=\sum_{k=0}^{d-1}\vert k+1\rangle\langle k\vert.
\]

Here, $P$ is a translation operator.

TUVU, Localizing matrix etc....

\subsubsection{Deriving the expression for fidelity}

Let $\left\{ \left\{ \bar{\Pi_{i}}\right\} _{i\in V},\vert\bar{\psi}\rangle\right\} $
be the ideal configuration and $\left\{ \left\{ \Pi_{i}\right\} ,\vert\psi\rangle\right\} $
be the candidate configuration. Our expression for the fidelity for
the aforementioned configurations is given by
\[
F\left(\alpha,\beta\right)=\alpha\sum_{i\in V}\langle\bar{\psi\vert}\bar{\Pi_{i}}\Pi_{i}\vert\psi\rangle+\beta\langle\bar{\psi\vert}\psi\rangle,
\]
where $\alpha$,$\beta\in\mathbb{R}$ and are weights on fidelitiy
for states and measured states respectively.

The total fidelity for state and projectors for odd $n$-cycle scenario
in the ideal case is $n+1.$

\subsubsection{Providing the lower bound on fidelity}

NPA analoge for theta body

\section{Experimental results}

The experimental implementation corresponding to the work in ref \cite{Bharti2018}
was carried on.

\section{Discussion}

\section{Conclusions}

\bibliographystyle{plain}
\bibliography{theta_number}

\end{document}

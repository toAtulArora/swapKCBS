%% LyX 2.3.5.2 created this file.  For more info, see http://www.lyx.org/.
%% Do not edit unless you really know what you are doing.
\documentclass[british]{article}
\usepackage[T1]{fontenc}
\usepackage[latin9]{inputenc}
\usepackage{geometry}
\geometry{verbose,lmargin=1cm,rmargin=1cm}
\usepackage{refstyle}
\usepackage{amsmath}
\usepackage{amsthm}
\usepackage{amssymb}

\makeatletter

%%%%%%%%%%%%%%%%%%%%%%%%%%%%%% LyX specific LaTeX commands.

\AtBeginDocument{\providecommand\Eqref[1]{\ref{Eq:#1}}}
\RS@ifundefined{subsecref}
  {\newref{subsec}{name = \RSsectxt}}
  {}
\RS@ifundefined{thmref}
  {\def\RSthmtxt{theorem~}\newref{thm}{name = \RSthmtxt}}
  {}
\RS@ifundefined{lemref}
  {\def\RSlemtxt{lemma~}\newref{lem}{name = \RSlemtxt}}
  {}


%%%%%%%%%%%%%%%%%%%%%%%%%%%%%% Textclass specific LaTeX commands.
\numberwithin{equation}{section}
\numberwithin{figure}{section}

\makeatother

\usepackage{babel}
\begin{document}
\title{Local and Robust Self testing using trapped ions}
\author{Xiaomin etc.., Atul Singh Arora, Kishor Bharti, Adan Cabello, Kwek
Leong Chuan}
\maketitle
\begin{abstract}
We provide experimental implementation of KCBS self-testing scheme.
Theoretical tools are supplemented to render the results in \textit{Physical
Review Letters 122 (25), 250403 }practical.
\end{abstract}

\section{Introduction}

\subsection{Why self-testing}

\subsubsection{State of the art}
\begin{itemize}
\item Bell/Device independence | Tensor structure
\begin{itemize}
\item Limitation: Difficult to enforce etc
\item Curve less robust to noise
\item Math is harder
\end{itemize}
\item Semi device independent
\begin{itemize}
\item Easier to implement
\item EPR steering {[}K{]}; almost like blind computing
\end{itemize}
\item All bipartite entangled states can be self-tested by quantum steering
| EPR steering
\end{itemize}

\subsubsection{Local Self testing }

Motivation: Device certification.
\begin{itemize}
\item Computation, which is usually local, it is inconvenient also less
secure to trust another party etc.
\end{itemize}
State of the art
\begin{itemize}
\item Vidick et al
\item Kishor's et al article on self-testing. Robustness curve was given
up to multiplicative constants. Not directly implementable. 
\end{itemize}

\subsubsection{Contextuality}

Informal quick introduction

\subsubsection{Contribution}

Address limitations of Kishor et al's article
\begin{itemize}
\item theoretically obtained the robustness curve
\begin{itemize}
\item Moment matrices based approach
\item First robustness curve for a non-contextuality inequality (five cycles
no less!)
\end{itemize}
\item experimental demonstration of local self-testing
\end{itemize}

\subsubsection{Assumptions and Memory leakage :P}

Self testing works under the following.

\subsubsection{Circumventing the memory assumption}
\begin{itemize}
\item Memory based
\begin{itemize}
\item Quantum classical gap?
\end{itemize}
\item Dimensions based measures
\begin{itemize}
\item Quantum classical gap 
\item log{[}classical dimensions{]}=memory; quantum dimension = quantum
memory
\end{itemize}
\end{itemize}
Criticism:

BQP$\subseteq$PSPACE; that's fine we hope to find the exact curve
to certify
\begin{itemize}
\item Poly separation or some such.
\end{itemize}

\section{Background}

\subsection{Exclusivity graph approach to contextuality}

\subsection{Robust self-testing}
\begin{itemize}
\item Definition
\item Statement for KCBS {[}PRL{]}
\item Robustness curve is missing! Or is it?
\end{itemize}


\section{Robustness Curve}

{[}DISCLAIMER: I am writing the following as rough notes which may
ramble but should at least be consistent; in the following iteration,
I hope to make improve the presentation{]}

What we have is some experimental data

\subsection{Description}

\global\long\def\KCBS{\text{KCBS}}%

\global\long\def\tr{\text{tr}}%

TODO:
\begin{itemize}
\item IID
\item Restricted subspace or full subspace, Bell
\item Parallel or Serial
\begin{itemize}
\item When IID, parallel and serial should become identical
\item When not IID, parallel should be more general
\end{itemize}
\item Clarify the issue with the sum
\end{itemize}
Consider the KCBS scenario, i.e. an experimental scenario which is
specified by the events $e_{1},e_{2}\dots e_{5}$ whose exclusivity
is given by a $5$-cycle exclusivity graph. Suppose we obtain the
probabilities $p_{1},p_{2}\dots p_{5}$ experimentally (and ensure
that they correspond to sharp measurements). We already saw that if
these values nearly saturate the quantum bound for the KCBS non-contextuality
inequality, i.e. $\sum_{i=1}^{5}p_{i}$ approaches $\sqrt{5}\approx2.236\dots$,
then we know that all quantum realisations corresponding to the experimentally
obtained probabilities $\{p_{i}\}_{i=1}^{5}$ are almost equivalent
to $\rho^{\text{KCBS}},\{\Pi_{i}^{\text{KCBS}}\}_{i=1}^{5}$, up to
a global isometry. To be more precise, consider any arbitrary quantum
realisation given by a pure state $\rho$ and rank one projectors
$\{\Pi_{i}\}$ such that $\sum_{i=1}^{5}p_{i}=2+\epsilon$. It was
shown in \cite{Bharti2018} that then, there exists an isometry $V$
such that 
\begin{equation}
\left\Vert V\Pi_{i}V^{\dagger}-\Pi_{i}^{\KCBS}\right\Vert _{F}\le\mathcal{O}(\sqrt{\epsilon})\label{eq:projKish}
\end{equation}
 for all $i\in\{1,2\dots5\}$ and $\left\Vert \rho-\rho^{{\rm KCBS}}\right\Vert _{F}\le\mathcal{O}(\sqrt{\epsilon})$
where $\left\Vert A\right\Vert _{F}:={\rm tr}\left(\sqrt{A^{\dagger}A}\right)$.
{[}Verify this:{]} This result is stronger than the self-testing results
which are usually stated for Bell non-locality, in the following sense.
Typically, one is only able to make statements about the action of
the projectors $\Pi_{i}$ on the state $\rho$ (see, for instance,
...). Despite this, without the constant hidden in \Eqref{projKish},
one cannot obtain a robustness curve and thus one cannot apply this
in practice. In this work, we remedy this problem, taking inspiration
from the Bell self-testing approach. To this end, we give a slightly
different statement: we show that
\begin{equation}
\sum_{i=1}^{5}\mathcal{F}(V\Pi_{i}\rho\Pi_{i}V^{\dagger},\Pi_{i}^{\KCBS}\rho^{\KCBS}\Pi_{i}^{\KCBS})+\mathcal{F}(V\rho V^{\dagger},\rho^{\KCBS})\ge f(\epsilon)\label{eq:lowerbound}
\end{equation}
 where $\mathcal{F}(A,B):={\rm tr}\sqrt{\left|A^{1/2}B^{1/2}\right|}$,
we only require $\Pi_{i}$ to be projectors and allow $\rho$ to be
a mixed state. The advantage is that we are able to express $f(\epsilon)$
as a hierarchy of semi-definite programmes and compute lower bounds
explicitly. While we state our result for the KCBS inequality, it
readily extends to the $n$-cycle scenario. {[}TODO: if $f$ cannot
be shown to be independent of the individual $p_{i}$s, then we must
skip it{]}. 


\subsection{Overall Strategy}

TODO
\begin{itemize}
\item Verify if we require that $\Pi_{i}\Pi_{j}=0$ or that $\tr[\Pi_{i}\Pi_{j}\rho]=0$.
$\tr[\Pi_{i}\Pi_{j}\Pi_{k}\rho]$ (by assumption; write clearly in
the previous section)
\end{itemize}
We may restate the aforesaid discussion more symbolically as lower
bounding the value of the following objective function: 
\[
F:=\min_{\rho,\{\Pi_{i}\}}\max_{V}\left[\sum_{i=1}^{5}\mathcal{F}(V\Pi_{i}\rho\Pi_{i}V^{\dagger},\Pi_{i}^{\KCBS}\rho^{\KCBS}\Pi_{i}^{\KCBS})+\mathcal{F}(V\rho V^{\dagger},\rho^{\KCBS})\right]
\]
where $\rho,\{\Pi_{i}\}_{i=1}^{5}$ is a quantum realisation\footnote{note that we assume $\Pi_{i}$ are projectors as the measurements
are assumed to be sharp experimentally} of $\{p_{i}\}_{i=1}^{5}$, $V$ is an isometry from $\mathcal{H}$
to $\mathcal{H}^{\KCBS}$, i.e. from the space on which $\rho,\{\Pi_{i}\}_{i=1}^{5}$
act/are defined to that where $\rho^{\KCBS},\{\Pi_{i}^{\KCBS}\}_{i=1}^{5}$
act/are defined. At the broadest level, the idea is to drop the maximization
over $V$ with a particular isometry $V$ which is expressed in terms
of $\rho,\{\Pi_{i}\}_{i=1}^{5}$. Then, as we shall see, the expression
for the fidelity appears as a sum of terms of the following form.
Let $w$ be a word created from the letters, $\{\mathbb{I},\Pi_{1},\Pi_{2}\dots\Pi_{5},\hat{P}\}$
with $\hat{P}^{\dagger}\hat{P}=\mathbb{I}$, $\Pi_{i}^{2}=\Pi_{i}$
and $\Pi_{i}\Pi_{j}=0$ if $(i,j)\in E(G)$, i.e. when $i,j$ are
exclusive\footnote{We introduced $\hat{P}$ for completeness; its role is explained later.}.
The fidelity is a linear combination of these words, i.e. $F=\min_{\left\{ \left\langle w\right\rangle \right\} }\sum_{w}\alpha_{w}\left\langle w\right\rangle $
where $\tr[w\rho]=:\left\langle w\right\rangle $, subject to the
constraint that $\{\left\langle w\right\rangle \}_{w}$ corresponds
to a quantum realisation. The advantage of casting the problem in
this form is that one can now construct an NPA-like hierarchy. The
idea is simple to state. Treat $\left\{ \left\langle w\right\rangle \right\} _{w}$
as a vector. Denote by $Q$ the set of all such vectors which correspond
to a quantum realisation (of $\left\{ p_{i}\right\} _{i=1}^{5}$).
It turns out that one can impose constraints on words with $k$ letters,
for instance. Under these constraints, denote by $Q_{k}$ the set
that is obtained. Note that $Q_{k}\supseteq Q$ for it may contain
vectors which don't correspond to the quantum realisation. In fact,
$Q_{k}$ can be characterised using semi-definite programming constraints
(which in turn means they are efficiently computable). Intuitively,
it is clear that $\lim_{k\to\infty}Q_{k}=Q$. Further, it is also
clear that $F=\min_{\{\left\langle w\right\rangle \}_{w}\in Q}\sum_{w}\alpha_{w}\left\langle w\right\rangle \ge\min_{\left\{ \left\langle w\right\rangle \right\} _{w}\in Q_{k}}\sum_{w}\alpha_{w}\left\langle w\right\rangle $
as we are minimising over a larger set on the right hand side. 

We can now 
\begin{itemize}
\item Regularizing the experimental statistics
\item Constructing the isometry
\item Deriving the expression for fidelity
\item Providing the lower bound
\end{itemize}
The goal is to provide a lower bound on fidelity for the experimental
results in section 2. The following steps are required:

\subsection{Regularizing the experimental statistics {[}TODO: see later if this
is still relevant{]}}

In reference \cite{Bharti2018}, authors assume that the measurement
operators follow cyclic compatibility structure. However, this need
not be true in the real life experimental scenario. Thus, the experimental
data may not belong to quantum set due to finite statistics and violation
of the cyclic compatibility structure. This demands a regularization
of the experimental data for the theoretical analysis. Here, regularization
means finding the statistics which is closest to our experimental
statistics and belongs to the quantum set. The set of quantum behaviours
forms a convex set, known as theta body. 

\subsection{Providing a lower bound on the fidelity from regularized measurement
statistics}

\subsubsection{Constructing the isometry}

A swap circuit between two qudit registers is given by $S=TUVU$ where

\[
T=\mathbb{I}\otimes\sum_{k}\vert-k\rangle\langle k\vert
\]

\[
U=\sum_{k=0}^{d-1}P^{k}\otimes\vert k\rangle\langle k\vert
\]

\[
V=\sum_{k=0}^{d-1}\vert k\rangle\langle k\vert\otimes P^{-k}
\]

and

\[
P=\sum_{k=0}^{d-1}\vert k+1\rangle\langle k\vert.
\]

Here, $P$ is a translation operator.

TUVU, Localizing matrix etc....

\subsubsection{Deriving the expression for fidelity}

Let $\left\{ \left\{ \bar{\Pi_{i}}\right\} _{i\in V},\vert\bar{\psi}\rangle\right\} $
be the ideal configuration and $\left\{ \left\{ \Pi_{i}\right\} ,\vert\psi\rangle\right\} $
be the candidate configuration. Our expression for the fidelity for
the aforementioned configurations is given by
\[
F\left(\alpha,\beta\right)=\alpha\sum_{i\in V}\langle\bar{\psi\vert}\bar{\Pi_{i}}\Pi_{i}\vert\psi\rangle+\beta\langle\bar{\psi\vert}\psi\rangle,
\]
where $\alpha$,$\beta\in\mathbb{R}$ and are weights on fidelitiy
for states and measured states respectively.

The total fidelity for state and projectors for odd $n$-cycle scenario
in the ideal case is $n+1.$

\subsubsection{Providing the lower bound on fidelity}

NPA analoge for theta body

\section{Experimental results}

The experimental implementation corresponding to the work in ref \cite{Bharti2018}
was carried on.

\section{Discussion}

\section{Conclusions}

\bibliographystyle{plain}
\bibliography{theta_number}

\end{document}
